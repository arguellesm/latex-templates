\documentclass[a4paper,10pt]{article}

% Español
%\usepackage[spanish,es-noindentfirst]{babel}
% Geometría
\usepackage[a4paper,top=45pt,left=120pt]{geometry}

% Paquetes
\usepackage[T1]{fontenc}        % Codificación 8-bit
\usepackage[scaled]{helvet}     % Fuente sans serif
\usepackage[table]{xcolor}      % Color
\usepackage{listings}           % Bloques de código
\usepackage[most]{tcolorbox}    % Cuadros para código
\usepackage{graphicx}           % Imágenes
\usepackage{titlesec}           % Secciones y subsecciones
\usepackage{eso-pic}            % Color de fondo
\usepackage{changepage}         % Márgenes del título
\usepackage{tabularx}           % Tablas full width
\usepackage{lipsum}             % Filler text

% Pies de imagen
\usepackage[font=footnotesize,labelfont={sc},labelsep=period]{caption}

% Colores del documento
\definecolor{berenjena}{HTML}{825E6A}
\definecolor{champan}{HTML}{DDC8C4}
\definecolor{onyx}{HTML}{3D3D3D}
\definecolor{ceniza}{HTML}{A1A2A6}
\definecolor{platino}{HTML}{EBEBEB}
\definecolor{perla}{HTML}{F7F7F7}

% Estilo de los bloques de código
\lstset{
    backgroundcolor=\color{perla},
    basicstyle=\linespread{0.9}\ttfamily\small\color{onyx},
    commentstyle=\color{ceniza},
    keywordstyle=\color{berenjena},
    stringstyle=\color{champan},
    tabsize=2,
    breaklines=true,
    showstringspaces=false,
    numbers=none,
}

% Cuadro para los bloques de código
\newtcblisting{code}[1]{%
    colframe=perla,
    colback=perla,
    arc=1mm,
    left=3pt,right=0pt,top=-5pt,bottom=-5pt,
    listing only,
    listing options={language=#1}
}

% Fragmentos de código en línea
\newtcbox{ci}{
    on line,
    fontupper=\ttfamily\small,
    boxsep=1pt,
    left=0pt,right=0pt,top=0pt,bottom=0pt,
    colframe=perla,colback=perla,
    highlight math style={enhanced}
}

% Título de ejercicio
\newtcbox{ejer}{
    sharp corners,
    fontupper=\sffamily\bfseries\small,
    boxsep=0pt,
    colframe=platino, colback=platino,
    capture=minipage,
    width=\textwidth
}

% Formato de párrafos
\setlength{\parindent}{0em}
\setlength{\parskip}{1em}
\renewcommand{\baselinestretch}{1.2}

% Formato de secciones y subsecciones {indice}{texto}
\titleformat{\section}
    {\color{berenjena}\sffamily\bfseries\fontsize{10}{10}}{\thesection}{1em}{}
\titleformat{\subsection}
    {\color{ceniza}\sffamily\fontsize{10}{10}}{\thesubsection}{1em}{}

% Columna con color y ancho 
\newcolumntype{g}{>{\sffamily}>{\small}>{\columncolor{perla}}>{\hsize=.4\hsize}>{\raggedright}X}
\newcolumntype{s}{>{\small}>{\hsize=.6\hsize}X}

% Nueva fila y dos líneas
\def\nr{\\\hline\hline}

% ------------------------------------------------
% Documento
% ------------------------------------------------

\begin{document}

% Título
% ---------

% Bloque de color
\AddToShipoutPictureBG*{
    \AtPageUpperLeft{
        \color{berenjena}
        \hspace*{-.1\paperwidth}
        \raisebox{-.24\paperheight}{\rule{1.1\paperwidth}{.24\paperheight}}
    }
}

\begin{adjustwidth}{-105pt}{0pt}
    % Logo e información UGR
    \begin{minipage}[c]{.22\textwidth}
    \center
    {\sffamily{\color{perla} 
        \includegraphics[scale=.08]{./logo.png}
        
        \scriptsize\itshape
        Long and descriptive college department name

        University of Somewhere
    }}
    \end{minipage}
    % Línea divisora
    \begin{minipage}[c]{.09\textwidth}
        \center
        \color{perla}\rule{1pt}{115pt}
    \end{minipage}
    % Título en sí
    \begin{minipage}[c]{.6\textwidth}
    \raggedright
    {\sffamily{\color{perla}
        {\textsc ASSIGNMENT 1}
        \vspace{5pt}

        {\Large\textbf{Eggplant: An analysis}}
        \vspace{10pt}

        \textsc{Vegetables 101}
        \vspace{1pt}

        \small
        Jane Doe

        \today
    }} 
    \end{minipage}
\end{adjustwidth}

\arrayrulecolor{white}

% Contenido
% -----------
\vspace{100pt}

\section*{Introduction}
\lipsum[1][1-5]

\section*{Eggplant care}

\ejer{1. Soil}
\begin{tabularx}{\textwidth}{gs}
    \lipsum[2][6] & \lipsum[1][6-10] \nr
    \lipsum[2][8] & \lipsum[2][1] \nr
\end{tabularx}

\ejer{2. Water}
\begin{tabularx}{\textwidth}{gs}
    \lipsum[2][7] & \lipsum[4][1-6] \nr
    \lipsum[2][10] & \lipsum[3][1-6] \nr
\end{tabularx}

\ejer{3. Climate}
\begin{tabularx}{\textwidth}{gs}
    \lipsum[5][1] & \lipsum[5][3] \nr
    \lipsum[5][2] & \lipsum[5][4-8] \nr
\end{tabularx}







\end{document}
